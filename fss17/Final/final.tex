\documentclass[sigconf, proceedings, 9pt]{acmart}
\settopmatter{printacmref=false} % Removes citation information below abstract
\renewcommand\footnotetextcopyrightpermission[1]{} % removes footnote with 
%conference information in first column
\usepackage{bigstrut}
\usepackage{amsmath}
\usepackage{balance}
\usepackage{dsfont}
\usepackage{booktabs}
\usepackage{multirow}
\usepackage[english]{babel}
\usepackage{blindtext}
\usepackage{enumitem}
\setlist{leftmargin=*}
\definecolor{Code}{rgb}{.12,.79,.17}
\definecolor{steel}{rgb}{0.4, 0.4,0.7}
\definecolor{Green}{rgb}{.12,.79,.17}
\usepackage{tikz}
\usepackage{calc}
\linespread{1.0}
\def\checkmark{\tikz\fill[scale=0.3](0,.35) -- (.25,0) -- (1,.7) -- (.25,.15) 
-- cycle;} 
\def\scalecheck{\resizebox{\widthof{\checkmark}*\ratio{\widthof{x}}{\widthof{\normalsize
 x}}}{!}{\checkmark}}
%\usepackage[table,dvipsnames]{xcolor}
\usepackage{colortbl}
\newcommand{\gray}{\rowcolor[gray]{.95}}
\pagenumbering{arabic}
\renewcommand{\i}{\item}
\newcommand{\bi}{\begin{itemize}}
\newcommand{\be}{\begin{enumerate}}
\newcommand{\ei}{\end{itemize}}
\newcommand{\ee}{\end{enumerate}}
\newcommand{\tion}[1]{\S~\ref{sect:#1}}
\newcommand{\eq}[1]{\S~\ref{eq:#1}}
\newcommand{\fig}[1]{Fig.~\ref{fig:#1}}
\setlength{\abovedisplayskip}{4pt}
\setlength{\belowdisplayskip}{4pt}
\def\bibfont{\small}

\begin{document}
\title{On Recommending Actionable Changes\\For Software Projects}
\author{Rahul Krishna}
\affiliation{NC State University}
\email{i.m.ralk@gmail.com}
%\acmDOI{}
%\acmISBN{}
\acmPrice{}
\acmConference[Fss'17]{}{Foundations of Software Science}{Fall 2017.}{}

\begin{abstract}
Newer generation of software analytics have placed significant emphasis on data 
driven decision making. These decisions are usually based on 
lessons that arise from within a particular project. Some times it is also 
possible to derive these decisions from across multiple projects. In the past, 
research efforts have led to the development of XTREE 
for generating a set of actionable plans within and across projects. We 
postulated that, each of these plans, if followed will improve the quality of 
the software project. Our previous work, however, culminated in an open 
question --- do developers make use of these plans? If so, to what extent?
This work is an attempt to answer these questions. Specifically, we compare the 
overlap between changes made by developers and those recommended by XTREE. To 
this end, we mined several versions of nine popular open source software 
projects. Our results show that recommendations offered by conventional XTREE 
overlaps to an extent of 50\% with changes undertaken by the developers. 
Modified versions of XTREE was developed to improve this overlap. And these 
offer over 75\% overlap.

\end{abstract}

\maketitle

\section{Introduction}
\label{sect:intro}

\section{Related Work}
\label{sect:related}

\section{Planning in Software Engineering}
\label{sect:planning}

\section{Experimental Setup}
\label{sect:expt}

\subsection{Datasets}
\label{sect:datasets}
\input{datasets_img.tex}
\begin{figure}[!t]
\scriptsize
\centering
\begin{minipage}{\linewidth}
\includegraphics[width=\linewidth]{datasets.png}
\caption{Framework for data generation}\label{fig:datasets1}
\end{minipage}
\begin{minipage}{\linewidth}
\resizebox{\linewidth}{!}{%
% Table generated by Excel2LaTeX from sheet 'Sheet1'
\begin{tabular}{c|l|r|r|r}
\hline
\multirow{2}[4]{*}{Community} & 
\multicolumn{1}{c|}{\multirow{2}[4]{*}{Dataset}} & \multicolumn{3}{c}{\# of 
instances} \bigstrut\\
\cline{3-5}      &       & \multicolumn{1}{r|}{\# Versions} & 
\multicolumn{1}{l|}{\# samples} & Bugs (\%) \bigstrut\\
\hline
\multirow{10}[2]{*}{Java OO} 
& Ant & \multicolumn{1}{r|}{7} & 14289   & 438 (56.01) \\
& Azureus   & \multicolumn{1}{r|}{20} & 25818  & 350 
(20.69) \\
& Ecplise   & \multicolumn{1}{r|}{17} & 978905   & 119 (16.90) \\
& Hibernate & \multicolumn{1}{r|}{38} & 20712 & 303 (17.32) 
\\
& JMeter   & \multicolumn{1}{r|}{8} & 25915  & 707 (51.31) \\
& JStock & \multicolumn{1}{r|}{10} & 59796  & 562 (20.19) \\
& JUnit & \multicolumn{1}{r|}{8} & 25855   & 260 (57.91) \\
& Lucene & \multicolumn{1}{r|}{12} & 58582   & 367 (57.43) \\
& Weka & \multicolumn{1}{r|}{8} & 29816  & 1806 (54.40) \\
\hline
\end{tabular}
}
\end{minipage}
\caption{The figure lists the defect datasets gathered for this project.}
\label{fig:datasets}
\end{figure}
\input{metrics.tex}

The defect dataset used in this study comprises a total of 9 datasets that were 
mined from GitHub. The projects primarily involve systems developed in JAVA. 
For these systems, we measure defects at class and method-level granularity. 

The procedure used to mine these datasets is illustrated in 
\fig{datasets1}. All the datasets are 
described in terms of the attributes of \fig{static_metrics}. In order to 
gather these datasets, we use the following steps:
\be
\item Find the most popular JAVA projects on GitHub. For these projects, clone 
them locally and generate a complete log of release dates and commit messages.
\item For each release, build the project and gather the CK-Metrics. We used 
the ckjm tool\footnote{https://github.com/mjureczko/CKJM-extended} for this.
\item Use the commit messages between each releases to identify ``bug-fixing'' 
commits.
\item For each ``bug-fixing'' commit, find the respective files that are 
changed and mark them as defect prior to the release.
\ee

The above process, when repeated for all the projects, generates the required 
datasets. In \fig{datasets}, we summarize all the datasets that were mined for 
experimentation. All these datasets records the number of known defects 
for each class using a post-release bug tracking system. The classes are 
described in terms of 20 OO metrics, including CK metrics and McCabes 
complexity metrics. 


\subsection{Evaluation Strategy}
\label{sect:procedure}
\begin{figure}
\centering
\includegraphics[width=\linewidth]{expt_design.png}
 \caption{Experimental design}\label{fig:design}
\end{figure}

Our experimental design is shown in \fig{design}. We divide the
project data  into two parts the \textit{train set} and the \textit{test test}.
The train set could either be data that is available locally within a project, 
or it could be data from the bellwether dataset. We partition the train set to 
build both a {\em planner} and a {\em  verification oracle}. 

The following paragraph shows our {\bf principle of separation} which, we 
assert, is necessary to verify this kind of research: 
\begin{quote}
	{\em The verification oracle should be built with completely different data to 
	the planner, i.e., the data used for training a planner and constructing the 
	verification oracle are disjoint sets.}
\end{quote}


In \tion{planners} we describe the planners and a way to translate these plans 
into actionable decisions. To assess the effect of acting on these plans, we 
propose a  \textit{verification oracle}. This is necessary because it can be 
rather difficult  to judge the  effects of applying the plan. They cannot be 
assessed just by a rerun of the test
suite for three reasons: (1) The defects were recorded by a post release bug 
tracking system. It is entirely possible it escaped detection by the existing 
test suite; (2) Rewriting test cases to enable coverage of all possible 
scenarios presents a significant challenge; and (3) It make take a significant 
amount of effort to write new test cases that 
identify these changes as they are made.

To resolve this problem, SE researchers such as
Cheng et al.~\cite{Cheng10}, O'Keefe et al.~\cite{OKeeffe08,OKeeffe07},
Moghadam~\cite{Moghadam2011} and Mkaouer et al.~\cite{Mkaouer14}
use a {\em verification oracle} that is learned separately
from the primary oracle. The verification oracles assesses
how defective the code is before and after some
code changes.
For their secondary verification oracle,
Cheng, O'Keefe, Moghadam and  Mkaouer et al. use the QMOOD hierarchical
quality model~\cite{Bansiya02}.
A shortcoming of QMOOD
is that quality models learned from other projects
may perform poorly when applied to new projects~\cite{localvsglobal}.

Hence, for this study, we  eschew
older quality models like QMOOD.

\subsection{Performance Measures}
\label{sect:evaluation}

\subsubsection{Effectiveness}
\label{sect:effectiveness}
For planning and construction of a verification oracle, we divide the
project data into two parts the \textit{train set} and the \textit{test test}.
The train set could either be data that is available locally within a project, 
or it could be data from the bellwether dataset. We further partition the train 
set to build both a {\em planner} and a {\em verification oracle}. Note that: 
{\em The verification oracle should be built with completely different data to 
the planner.}

After constructing the planner and verification oracle, we (1)~deploy the 
{planner} to recommend plans; (2)~alter the {\em test} data according to these 
plans;
then (3)~apply the {verification oracle} to the altered data to estimate 
defects; then (3)~Compute the percent improvement, denoted by the following 
equation:
\begin{equation}
\small
\label{eq:effectiveness}
R=(1-\frac{\mathit{after}}{\mathit{before}})\times100\%
\end{equation}
The value of the measure $R$ has the following properties: (1) If $R = 0\%$, 
this means  ``no change from baseline''; 
(2) If $R > 0\%$, this indicates ``improvement'';
(3) If $R < 0\%$, this indicates ``optimization failure''. Ideally, an 
effective planner should have an improvement of $R>0$, where larger values 
indicate better performance. 

\subsection{Statistics}
\label{sect:stats}
Changes made to module in a software project are subject to inherent 
randomness. Researchers have endorsed the use of repeated runs 
to gather reliable evidence~\cite{vaux2012replicates}. Thus, we repeat the 
whole experiment independently using a moving window over 
the various versions  of the project (see \tion{procedure}) to provide evidence 
that the results are reproducible. These repeats provide us with a sufficiently 
large sample size to statistically compare the performances. 

At each position of the moving window, we collect the values of effectiveness R 
(\eq{effectiveness})
and overlap O (\eq{overlap}). We refrain from 
performing a cross validation because the process tends to mix the samples 
from training data (the source) and the test data (other target 
projects), which defeats the purpose of this study.

To rank the numbers collected above, we use the Scott-Knott test 
recommended by Mittas and Angelis~\cite{mittas13}. Scott-Knott is a 
top-down clustering approach used to rank different treatments. If that 
clustering finds an statistically significant splits in data, then some 
statistical test is applied to the two divisions to check if they are 
statistically significant different. If so, Scott-Knott recurses into both 
halves.

To  apply Scott-Knott, we sorted a list of  $l$ values of \eq{G} 
values found in $ls$ different methods. Then, we split $l$ into 
sub-lists $m,n$ in order to maximize the expected value of differences in 
the observed performances before and after divisions. E.g. for lists 
$l,m,n$ of size $ls,ms,ns$ where $l=m\cup n$: 
\[E(\Delta)=\frac{ms}{ls}abs(m.\mu - l.\mu)^2 + \frac{ns}{ls}abs(n.\mu - 
l.\mu)^2\] 


We then apply a statistical hypothesis test $H$ to check
if $m,n$ are significantly different. In our case, the conjunction of 
bootstrapping and A12 test. Both the techniques are non-parametric in nature, 
i.e., they do not make gaussian assumption about the data. As for hypothesis 
test, we use a non-parametric bootstrapping test as endorsed by Efron \& 
Tibshirani~\cite[p220-223]{efron93}. Even with statistical significance, it is 
possible that the difference
	can be so small as to be of no practical value. This is known as a ``small 
	effect''. To ensure that the statistical significance is not due to ``small 
	effect'' we use effect-size tests in conjunction with hypothesis tests. A 
	popular effect size test used in SE literature is the A12 test. It has been 
	endorsed by several SE researchers~\cite{leech2002call, poulding10, arcuri11, 
	shepperd12a, kampenes07, Kocaguneli2013:ep}. It was first proposed by Vargha 
	and Delany~\cite{vargha2000}. In our context,
	given the performance measure 
	G, the A12 statistics measures the
	probability that one treatment yields higher $G$ values than another. If the 
	two algorithms are equivalent, then A12 = 0.5. Likewise if $A12 \ge 0.6$, then 
	60\% of the times, values of one treatment are significantly greater that the 
	other. In such a case, it can be claimed that there is \textit{significant 
	effect} to justify the hypothesis test $H$. In our case, we divide the data if 
	\textit{both} bootstrap sampling and effect size test agree that a division is 
	statistically significant (with a
	confidence of 99\%) and not a small effect ($A12 \ge 0.6$). Then, we recurse 
	on each of these splits to rank G-scores from best to worst.



\section{Results}
\label{sect:results}

	\section{Reliability and Validity of Conclusions}

\section{Conclusions and Future Work}
\label{sect:conclusion}

\label{sect:valid}



The results of this paper are biased by our choice of code reorganization
goal (reducing defects) and our choice of measures collected from software
project (OO measures such as depth of inheritance, number of child classes,
etc). That said, it should be possible extend the methods of this paper to 
other
kinds of goals (e.g. maintainability, reliability, security, or the 
knowledge sharing
measures) and other kinds of
inputs (e.g. the process measures favored by Rahman,
Devanbu et al.~\cite{Rahman2013})

\subsection{Reliability}
Reliability refers to the consistency of the results obtained
from the research. It has at least two components: internal
and external reliability.

Internal reliability checks if an independent researcher
reanalyzing the data would come to the same conclusion.
To assist other researchers exploring this point, we offer a full 
replication package for this study at
https://github.com/rahlk/FSS17.

External reliability assesses how well independent researchers
could reproduce the study. To increase external
reliability, this paper has taken care to clearly define our
algorithms. Also, all the data used in this work is available
online.

For the researcher wishing to reproduce our work to other kinds of goals, 
we offer the following advice:

\begin{itemize}
\item Find a data source for the other measures of interest;
\item Implement another secondary verification oracle that can assess 
maintainability, reliability, security, technical debt, etc;
\item Implement a better primary verification oracle that can do ``better'' 
than XTREE at finding changes (where ``better'' is defined in terms
of the opinions of the verification oracle). 
\end{itemize}


\subsection{Validity}

This paper is a case study that studied the effects of  limiting 
unnecessary code reorganization on some data sets. This section discusses 
limitations of such case studies. In this context, validity refers to the 
extent to which a piece of research actually
investigates what the researcher purports to investigate.
Validity has at least two components: internal and
external validity.


Based on the case study presented above,
as well as the discussion in \tion{prelim},
we believe that defect indicators (e.g. \mbox{{\em loc}$>$ 100})
have limited external validity beyond the projects from which they are 
derived.
While specific models are externally valid,
there may still be general methods like XTREE for finding the good local 
models.

Our definition of bad smells is limited to those represented by OO code 
metrics (a premise often used in related work).
XTREE, Shatnawi, Alves et al. can  only comment
on bad smells   expressed as code metrics
in the historical log of a project.

If developers want to justify their code reorganizations
via bad smells expressed in other terminology,
then the  analysis of this paper must:
\begin{itemize}
	\item Either wait till
	data about those new
	terms has been collected.
	\item Or, apply cutting edge transfer learning
	methods~\cite{Nam15,Jing15, krishna16} to map data from other projects
	into the current one.
\end{itemize}
Note that the transfer learning approach would
be highly experimental and require more study
before it can be safely recommended.

Sampling bias threatens any data mining analysis; i.e., what matters
there may not be true here. For example, the data sets used here comes 
from 
Jureczko et al. and any biases in their selection procedures
threaten the validity of these results.
That said,
the best we can do is define our methods and publicize our data and code so 
that other researchers can
try to repeat our results and, perhaps, point out a previously unknown bias
in our analysis. Hopefully, other researchers will emulate our methods in
order to repeat, refute, or improve our results.

\balance
\bibliographystyle{ACM-Reference-format}
\bibliography{final}
\end{document}
